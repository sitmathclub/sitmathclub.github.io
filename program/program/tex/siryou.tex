%数理科学研究会
%資料用テンプレート
\documentclass[titlepage, a4paper, 11pt, dvipdfmx]{jsarticle}
\usepackage{otf} %otfパッケージを読み込むことで様々な問題を回避できる
\usepackage[T1]{fontenc} %T1エンコードにすることで様々な問題を回避できる
\usepackage{lmodern} %フォントサイズの問題を回避
\usepackage{exscale} %大型演算子の問題を回避
\usepackage{etex}
\usepackage[dvipdfm, truedimen, top=25truemm, hmargin=25truemm, bottom=25truemm]{geometry} %用紙サイズの設定
\和暦 %日付を和暦に

%スタイルファイルを追加したい場合は以下のように書く
\usepackage{amsmath,amssymb}

\title{\Huge タイトル}
\date{\today}
\author{\Large XY99999 \quad 芝浦 太郎}
\begin{document}
\maketitle
\pagenumbering{roman}
\tableofcontents
\newpage
\pagenumbering{arabic}


%%%%%%%%%%%%%%%%%%%%%%%%%%%%%%%%%%%%%%%%%%%%%%%%%%%%%%%%%%%%%%%%%%%%%%%%%%%%%
%%%%%%%%%%%%%%%%%%%%%%%%%%%%%%%%%%%%%%%%%%%%%%%%%%%%%%%%%%%%%%%%%%%%%%%%%%%%%
%%%%%%%%%%%%%%%%%%%%%%%%%%%%%%%%%%%%%%%%%%%%%%%%%%%%%%%%%%%%%%%%%%%%%%%%%%%%%
%以下サンプル%ここから書き始めてください.
\newcommand{\TQBF}{The quick brown fox jumps over the lazy dog. }
\section{A}
\TQBF \TQBF \TQBF \TQBF \TQBF \TQBF \TQBF \TQBF
\subsection{a}
\TQBF \TQBF \TQBF \TQBF \TQBF \TQBF \TQBF \TQBF
\TQBF \TQBF \TQBF \TQBF \TQBF \TQBF \TQBF \TQBF
\TQBF \TQBF \TQBF \TQBF \TQBF \TQBF \TQBF \TQBF
\subsection{b}
\TQBF \TQBF \TQBF \TQBF \TQBF \TQBF \TQBF \TQBF
\TQBF \TQBF \TQBF \TQBF \TQBF \TQBF \TQBF \TQBF
\TQBF \TQBF \TQBF \TQBF \TQBF \TQBF \TQBF \TQBF
\TQBF \TQBF \TQBF \TQBF \TQBF \TQBF \TQBF \TQBF
\subsection{c}
\TQBF \TQBF \TQBF \TQBF \TQBF \TQBF \TQBF \TQBF
\TQBF \TQBF \TQBF \TQBF \TQBF \TQBF \TQBF \TQBF
\TQBF \TQBF \TQBF \TQBF \TQBF \TQBF \TQBF \TQBF
\TQBF \TQBF \TQBF \TQBF \TQBF \TQBF \TQBF \TQBF
\TQBF \TQBF \TQBF \TQBF \TQBF \TQBF \TQBF \TQBF
\TQBF \TQBF \TQBF \TQBF \TQBF \TQBF \TQBF \TQBF

\section{B}
\TQBF \TQBF \TQBF \TQBF \TQBF \TQBF \TQBF \TQBF
\TQBF \TQBF \TQBF \TQBF \TQBF \TQBF \TQBF \TQBF
\TQBF \TQBF \TQBF \TQBF \TQBF \TQBF \TQBF \TQBF
\TQBF \TQBF \TQBF \TQBF \TQBF \TQBF \TQBF \TQBF
\TQBF \TQBF \TQBF \TQBF \TQBF \TQBF \TQBF \TQBF
\TQBF \TQBF \TQBF \TQBF \TQBF \TQBF \TQBF \TQBF
\TQBF \TQBF \TQBF \TQBF \TQBF \TQBF \TQBF \TQBF
\TQBF \TQBF \TQBF \TQBF \TQBF \TQBF \TQBF \TQBF
\TQBF \TQBF \TQBF \TQBF \TQBF \TQBF \TQBF \TQBF
\TQBF \TQBF \TQBF \TQBF \TQBF \TQBF \TQBF \TQBF
\TQBF \TQBF \TQBF \TQBF \TQBF \TQBF \TQBF \TQBF
\TQBF \TQBF \TQBF \TQBF \TQBF \TQBF \TQBF \TQBF
\TQBF \TQBF \TQBF \TQBF \TQBF \TQBF \TQBF \TQBF
\TQBF \TQBF \TQBF \TQBF \TQBF \TQBF \TQBF \TQBF

\section{C}
\TQBF \TQBF \TQBF \TQBF \TQBF \TQBF \TQBF \TQBF
\TQBF \TQBF \TQBF \TQBF \TQBF \TQBF \TQBF \TQBF
\TQBF \TQBF \TQBF \TQBF \TQBF \TQBF \TQBF \TQBF
\TQBF \TQBF \TQBF \TQBF \TQBF \TQBF \TQBF \TQBF
\TQBF \TQBF \TQBF \TQBF \TQBF \TQBF \TQBF \TQBF
\TQBF \TQBF \TQBF \TQBF \TQBF \TQBF \TQBF \TQBF
\TQBF \TQBF \TQBF \TQBF \TQBF \TQBF \TQBF \TQBF
\TQBF \TQBF \TQBF \TQBF \TQBF \TQBF \TQBF \TQBF
\TQBF \TQBF \TQBF \TQBF \TQBF \TQBF \TQBF \TQBF
\TQBF \TQBF \TQBF \TQBF \TQBF \TQBF \TQBF \TQBF
\TQBF \TQBF \TQBF \TQBF \TQBF \TQBF \TQBF \TQBF
\TQBF \TQBF \TQBF \TQBF \TQBF \TQBF \TQBF \TQBF
\TQBF \TQBF \TQBF \TQBF \TQBF \TQBF \TQBF \TQBF
\TQBF \TQBF \TQBF \TQBF \TQBF \TQBF \TQBF \TQBF
\begin{thebibliography}{1}
    %著者, タイトル, 出版社, 現在の版の第1刷が発行された年.
    \bibitem{TeX} Donald E.Knuth 著, 鷺谷好輝 訳, \TeX ブック コンピュータによる組版システム, アスキー, 1989.
    \bibitem{LaTeX} 奥村晴彦, 黒木祐介, [改訂第7版] \LaTeXe 美文書作成入門, 技術評論社, 2017.
\end{thebibliography}
\end{document}
